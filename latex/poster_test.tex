%\documentclass[final]{beamer}
\documentclass[10pt, ]{beamer}
%\documentclass[10pt, handout]{beamer}
\usepackage[orientation=portrait,
            size=a0,
            scale=1.25
            ]{beamerposter}
%\usepackage{markdown} 
\usepackage[utf8]{inputenc}
\usepackage{multicol}
\usepackage{hyperref}
\usepackage{graphicx}
\usepackage{color}
\usepackage{framed}
\usepackage{subcaption}
\usepackage{float}
\graphicspath{ {images/} }
\usepackage{blindtext}
\usepackage{mathtools}
\usepackage{amssymb}
\usepackage{amsfonts}
\usepackage{amsthm}
\usepackage[backend=biber]{biblatex}
\addbibresource{mybib.bib}

%\usepackage[a4paper, width=150mm, top=25mm, bottom=25mm]{geometry}
%\usepackage{fancyhdr}
%\setlength{\headheight}{12pt}
%\pagestyle{fancy}
%\pagestyle{empty}
\usepackage{lipsum}


\title{Spam Beamer}
\subtitle{boring slides about spam}
\author{Kolb, Yiftach}
\date{\today}
\institute[FU and MPG]{
\centering
\vfill
{\includegraphics[width=0.8\linewidth]{images/MPIMG_RGB_gruen.png}}\\
\vfill
{\includegraphics[width=0.8\linewidth]{images/fu-logo_bildschirm_RGB1.jpg}}
}

\date{\today}


\begin{document}


\maketitle

%\begin{frame}[t]

%\begin{frame}

%\begin{multicols}{2}


\begin{frame}
\frametitle{Spam}

\begin{multicols}{3}

\section{Intro}
Spam~\footfullcite{russkikh2020style} is good.
%\end{frame}

\lipsum[0]

%\begin{frame}
And now to something~\cite{guo2017improved} completely different \dots

foo ...
foo ...
foo ...

foo ...
foo ...
foo ...

\section{foo}
\lipsum[1]
foo ...
foo ...
foo ...

\lipsum[2]

\section{Reference}
%\printbibliography

foo ...
foo ...
foo ...

\end{multicols}
\end{frame}
\end{document}
