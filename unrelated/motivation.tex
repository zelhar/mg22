\documentclass[a4paper,10pt]{article}
\usepackage{enumerate}
\usepackage{verbatim}
\usepackage{lipsum}
\usepackage{setspace}
\usepackage[backend=biber]{biblatex}
\addbibresource{bibfile.bib}
\usepackage{hyperref}
\usepackage{graphicx}

\setlength{\parskip}{1.0em}
\setlength{\parindent}{1em}

\def\shrug{\texttt{\raisebox{0.75em}{\char`\_}\char`\\\char`\_\kern-0.5ex(\kern-0.25ex\raisebox{0.25ex}{\rotatebox{45}{\raisebox{-.75ex}"\kern-1.5ex\rotatebox{-90})}}\kern-0.5ex)\kern-0.5ex\char`\_/\raisebox{0.75em}{\char`\_}}}

\begin{document}

\setstretch{1.2}

\title{Letter of Motivation}
\author{Yiftach Kolb}
\date{\today}

\maketitle

I started my academic studies at the Technion on a physics and mathematics
program and coming into the studies I was leaning towards the
physics side, which was also a top interest of mine growing up.
However that changed in the course of the first year and a half of my studies.
I didn't like labs nor the fact that our mathematical knowledge was severely
lagging behind the methods we were using in the physics courses---
things like Lagrange
multipliers, differential equations and integration methods.
So I switched to pure math and even within math, I was more interested in
abstract algebra, set theory, logic and combinatoric rather than for example,
numerical analysis and PDEs.

Afterwards although I left the university I didn't leave math and kept studying
here and there, but I picked up additional interests. Eventually when I decided
to get back to school I chose bioinformatics specifically because I
wanted to deal with life science this time and I knew that my strength lies more
on the math, algorithms and programming part of the business than the bio.
As mentioned back in the day I had a
need to explore everything from the ground up and it was very difficult for me
to use any technique if I didn't personally knew the underlying theory and
was convinced in its correctness.
This time around that is no longer an issue. I will
happily utilize anything as a black box as long as it does the work I need it
for. Well it's not exactly like that I still try to learn everything but I can
prioritize better for sure.

And now I've reached a point in my formal education where I need to narrow down
my focus further.
In the course of my master studies I became very interested and very involved
with neural networks and machine learning~\cite{mpgvaeRepo}. My thesis deals with a particular VAE
model.
I am by no means an expert on the topic but I want to learn more.
There are some big and bigger themes that I am curious about (probably too naively):

\begin{itemize}
\item{} Developing neural network architectures that are suitable for a specific
class of datasets.
\item{} What are the operating principles of real neural networks (nerves), can
they be simulated and imitated. 
\item{} Machine--nerve interface. Is it feasible for example to create an
artificial retina.
\item{} Can concepts of thinking, abstraction, perception, intelligence etc. be
mathematically formalized and implemented by a computer program.
\end{itemize}

So why apply to ISTA? 
From what I see you have an interdisciplinary
research groups and neuroscience and machine learning is one of the major topics
at your fine establishment. 
It was actually difficult to choose only five professors for my application form
because there are at least two or three additional research groups which I could
see myself applying to.
It sounds very interesting and it matches closely what I envision, for the type
of research that I want
to be doing. And for that to happen, I am willing to downgrade my location from
super cool Berlin to somewhat more boring Vienna (or its vicinity rather).

\nocite{mpgvaeRepo}
\nocite{mg22Repo}
\printbibliography

\end{document}
